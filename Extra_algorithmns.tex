\begin{algorithm}
\caption{Barabási-Albert (BA) Model Algorithm}\label{alg:BA}
\begin{algorithmic}[1]
\State \textbf{Input:} Initial number of nodes $m_0$, number of edges to attach $m \leq m_0$, total number of nodes $N$
\State \textbf{Output:} Scale-free network with $N$ nodes
\State

\State \textbf{Step 1:} \textbf{Initialization}
\State Create an initial network with $m_0$ nodes and connect every node to every other node (complete graph).

\State \textbf{Step 2:} \textbf{Growth}
\For{$i = m_0 + 1$ to $N$}
    \State Add a new node $i$ with $m$ edges.
    \State Connect the new node $i$ to $m$ existing nodes chosen with probability proportional to their degree.
    \For{each existing node $j$}
        \State Calculate the probability $\Pi(j)$ that node $i$ connects to node $j$:
        \[
        \Pi(j) = \frac{k_j}{\sum_{l} k_l}
        \]
        where $k_j$ is the degree of node $j$ and the sum is over all existing nodes $l$.
    \EndFor
\EndFor

\State \textbf{Step 3:} \textbf{Repeat}
\State Repeat the growth process until the network contains $N$ nodes.

\State \textbf{Step 4:} \textbf{Output the network}
\State Return the generated scale-free network.
\end{algorithmic}
\end{algorithm}


\subsubsection{Graph Isomorphism}
Given two graphs 

\begin{equation}
    (u,v) \in E_1 \Leftrightarrow (f(u),f(v)) \in E_2
\end{equation}

\subsubsection{Degree Correlation}

\begin{equation}
    r = \frac{\sum(x_i - \hat{x})(y_i-\hat{y}}{\sqrt{\sum}(x_i-\hat{x})^2\sum(y_i-\hat{y})^2}
\end{equation}


%\subsubsection{Spectral Comparison}
%\subsubsection{Clustering Coefficient Comparison}

OLD DATA ANALYSIS
% Data collection
The data will be collected through experimentation in a virtual network lab, as outlined above. Measurement of each configuration will be repeated five times, in order to ensure the validity of the results and allow for statistical analysis. 

% Data preparation
The network topology structure is pre-defined inside a YAML configuration file. Whereby the topology is defined in a set of \textit{nodes} and the \textit{links} between given nodes. Each node has a given \textit{kind} and \textit{image}, with the kind defining the node configuration and behaviour. Each node also has an \textit{image}, which defines the operating system which is to be ran on a given node. Each \textit{link} is defined through an array of end-points between each node. 

% Tools/Mathimatical models used for analysis
Experimental data will be exported in JSON format, from which it can be easily processed. The mean, mode, median and standard deviation will be presented for each finding. Analysis of the implemented model will be carried out using several metrics in order to illustrate comparisons between it's performance and the original model's performance. Several plot types such as scatter plots, line plots and violin plots will be used to visualize; negative and positive correlations and also to represent the standard deviation of results respectively. Furthermore, tables of data will also be used to provide accurate numerical data in a straightforward manner.