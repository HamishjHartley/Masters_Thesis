\section{Recommendations and conclusions}

\subsection{Recommendations}
%Provide recommendations to practice
Based on on the results of the evaluation of the vary topology models, several recommendations can be made for implementation in a virtual network lab. Firstly, it is important to establish the intent and context in which the virtual lab will be used, as this will relate directly to the recommendations given. If the context of the virtual lab is to ensure a highly realistic environment, then it would be recommended to use network topologies which exhibit a fairly high level of associativity, with a degree correlation coefficient less than -0.2. This characteristic was observed in the distribution of degree correlation coefficient, across 28 real-world networks. Furthermore, if one is inclined to use randomly generated topologies which most closely capture real-world structures, the Barabasi-Albert model using $m=3$ would be best suited out of the evaluated stochastic topology generation methods.

For other contexts, such as performing a comprehensive analysis of how a scanning method performs, implementing an enumeration of diverse topology networks which exhibit variance in both size and structure is recommended. In other words, in such a context no topology is preferential. By using a varied set of topologies, with randomly generated origins such as the Erdos-Renyi and Barabasi-Albert models in conjunction with real-world examples the available test coverage is increased. 

Regardless of the context, the proposed virtual network lab could be integrated directly into a unit testing frame-work, via an automated pipeline. By leveraging the straightforward deployment of lab sessions via containerlab, a developer or team could have verifiable proof that the tool meets specified requirements and executes in an intended manner. This would be particularly useful if done in a native Linux environment or using Window Subsystem for Linux\cite{wsl} to ensure a smooth workflow.

\subsection{Conclusion}
% robust and clear description of the advancement in knowledge and situate that advance within the context of what is already known within the discipline


%Provide a critical discussion that links the findings back tothe original review
In conclusion, several stochastic topology generation methods have been evaluated and compared with real-world topologies. Through this evaluation, it was established that the Barabasi-Albert model with parameter $m=3$ was the most similar model to the real-world topologies from the Internet Topology Zoo \cite{topology_zoo}. This was due to both the Barabasi-Albert model and real-world networks being observed to be assortative, favouring new links towards nodes with higher degrees, also known as "hubs". Therefore, it was proposed that if the intended context was to closely replicate a real-world scenario, the Barabasi-Albert model with parameter $m=3$ would be the best suited. However, it was also concluded that the Erdos-Renyi model still could provide some benefit by increasing the variety of possible topology configurations. 

A virtual network lab environment utilising the containerlab\cite{containerlab} framework, and NetworkX \cite{networkX} python library has been proposed, for the intended purpose of implementing the topology structures. Furthermore, a pipeline solution comprising of these elements has been proposed and partially implemented, with code written to parse a real-world or generated topology to a yml file in the format required for ContainerLab to deploy correctly. If fully implemented, this pipeline can be leveraged for a potentially faster workflow and allow for a robust testing framework to be established, possibly incorporating automatic unit testing. The need for an effective testing environment which allows for varying topology configurations has been discussed at length in the original literature review; whilst contemporary internet scanning methods have improved on previous approaches there are still many possible improvements which could be made. Through the utilisation of an optimised and streamlined testing environment, these improvements can be more readily achieved. Furthermore, being able to fully control the nature and characteristics of such an environment ensures the replicability and validity of yielded results, allowing testing to be carried out as many times as required. 

% outline major achievements, barriers to its success and particularly innovative aspects
 However, further work is needed to establish the suitability of generated topology models. Implementation of the evaluated topologies in the proposed virtual lab has not been covered in this paper. This is due to several reasons, including the lack of required resources available to simulate topologies of a large size. This is a drawback of the proposed network lab, which is also directly related to one of it's main advantages; it is a highly realistic environment which simulates containerised operating systems found in real-world networks, consequently this means a high resource requirement in order run such environments. A further barrier which impeded progress was interacting with the different networking oriented operating systems used on routers in the virtual lab. As they are highly specialized operating systems, they often had differing and non-standard command sets which caused difficulty, especially when changing between multiple routers hosting different operating systems in the network. Expanding from this point, the varying package managers and repositories with which to access the packages from, such as the Alpine Package Manager (apk)\cite{alpine} were also different from the ubiquitous "apt" used on Debian based Linux distributions. 

 Another suggested branch of enquiry in future would be to explore a wider variety of topology generation methods, possibly implementing the topology library approach previously mentioned. Combining Machine learning and Artificial Intelligence at any stage of the process could also also likely could be a fruitful avenue of research. This includes generation of topology structures using a re-enforcement learning approach, based on a specified training set to optimise for the desired parameter, it also could be applied to optimise performance of network scanning tools possibly using the previously discussed Zeph algorithm \cite{zephMap}. 

 To finish, it has been established that topology structures when implemented in an appropriate virtual lab context, are of central importance. Utilising both real-world and generated topologies can provide varying structures in which network scanning tools can be rigorously evaluated; which could possibly lead to further improvements of such methods in the future. 